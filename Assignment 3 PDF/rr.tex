\documentclass[]{article}
\usepackage[UKenglish]{babel}
\usepackage{fixltx2e}
\usepackage{multicol}
\usepackage{amsmath}
\usepackage{amsfonts}
\usepackage{float}
\usepackage[margin=1in]{geometry}
\usepackage{graphicx}

\title{\textbf{Software Design III 3BB4} \\ \textit{Assignment III}}
\date{March 17, 2015}
\author{Jadees Anton {\ } Connor Hallett {\ } Spencer Lee {\ } Nicolas Lelievre \\ \textit{1213386 {\ }{\ }{\ }{\ }{\ }{\ }{\ }{\ } 1158083 {\ }{\ }{\ }{\ }{\ }{\ }{\ }{\ } 1224941 {\ }{\ }{\ }{\ }{\ }{\ }{\ }{\ } 1203446}}

\begin{document}
\maketitle
\setlength{\pdfpagewidth}{8.5in}
\setlength{\pdfpageheight}{11in}

\section*{Round Robin Scheduler}
This document elaborates on how the \verb|rr.lts| LTSA works to schedule a series of processes using round-robin scheduling.
	
\subsection*{GENERATOR}
GENERATOR=GENERATOR[0],\\
GENERATOR[num:IF]=(enterarrq[num]->GENERATOR[(num+1)%MaxIF]).\\
\\
\\Generator creates processes to be executed by the CPU. It enqueues them as long as there is room for a new procress in the queue.\\\
\\
\textbf{ Design Decision 1: Processes to be executed have runtimes that are later defined by the system.} 
\\
\\GENERATOR pushes the processes to the queue in a first-in-first-out (FIFO) manner. As per the requirements outlined by the assignment,  processes are immediately enqueued but not immediately loaded into the CPU. 

\subsection*{QUEUE}
QUEUE holds processes, generated by GENERATOR, that are ready for execution by the CPU. \\
\\
\textbf{Design Decision 2: The QUEUE will have space for 6 processes that are ready for execution.}
\\
\\Our queue holds and enqueues the processes waiting for execution in a FIFO manner. 

\subsection*{CPU}
CPU takes the first process from QUEUE that is ready to be executed and loads it, similar to how DISPATCHER would function. It is here that the user dictates how long each process executes for.\\
\\
\textbf{Design Decision 3: Any process that has been selected to execute will run until completion - the process will not be re-queued at the end of the queue. \\
\\Design Decision 4: CPU also contains the functionalities of DISPATCHER and GRIMREAPER.}
\\
\\Once a process has completed its execution, the CPU unloads it, similar to what GRIMREAPER would do. 
\subsection*{RR SCHEDULER}
RR_SCHEDULER is a parallel composition of the above three processes.
\end{document}

